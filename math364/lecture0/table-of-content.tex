% !TEX program = xelatex

\documentclass[a4paper, 12pt]{article}
\usepackage[margin=0.75in, top=1in]{geometry}
\usepackage{amsmath}
\usepackage{polyglossia}
\setmainlanguage{arabic}
\setotherlanguage{english}
\setmainfont{Rubik}
%\usepackage{arabtex}
%\usepackage{utf8}
%\setcode{utf8}
\pagenumbering{gobble}
\usepackage{tocloft} % Add the tocloft package


% Define the style for the table of contents entries
\renewcommand{\cftsecpagefont}{}
\renewcommand{\cftsubsecpagefont}{}

\begin{document}
    \tableofcontents
    \section{المتطابقات المثلثية والمعادلات المثلثية}
        \subsection{المتطابقات المثلثية}
        \subsection{إثبات صحة المتطابقات المثلثية}
        \subsection{المتطابقات المثلثية لمجموع زاويتين والفرق بينهما}
        \subsection{المتطابقات المثلثية لضعف الزاوية ونصفها}
        \subsection{حل المعادلات المثلثية}
    \section{تحليل الدوال}
        \subsection{الدوال}
        \subsection{تحليل التمثيل البياني للدوال والعلاقات}
        \subsection{الاتصال وسلوك الدالة عند طرفي تعريفها، النهايات}
        \subsection{القيم القصوى ومعدل التغير}
        \subsection{الدوال الأصلية والتحويلات}
        \subsection{العمليات على الدوال وتركيبها}
        \subsection{العلاقات العكسية والدوال}
    \section{النهايات والاشتقاق}
        \subsection{تقدير النهايات بيانيا}
        \subsection{حساب النهايات جبريا}
        \subsection{الاشتقاق}
        \subsection{المماس وسرعة المتجه}
\end{document}